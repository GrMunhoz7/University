\documentclass[
	12pt,				% tamanho da fonte
	openright,			% capítulos começam em pág ímpar (insere página vazia caso preciso)
	oneside,			% para impressão em recto e verso. Oposto a oneside
	a4paper,			% tamanho do papel. 
	english,			% idioma adicional para hifenização
	french,				% idioma adicional para hifenização
	spanish,			% idioma adicional para hifenização
	brazil				% o último idioma é o principal do documento
	]{abntex2}

\usepackage{lmodern}			% Usa a fonte Latin Modern			
\usepackage[T1]{fontenc}		% Selecao de codigos de fonte.
\usepackage[utf8]{inputenc}		% Codificacao do documento (conversão automática dos acentos)
\usepackage{indentfirst}		% Indenta o primeiro parágrafo de cada seção.
\usepackage{color}				% Controle das cores
\usepackage{graphicx}			% Inclusão de gráficos
\usepackage{microtype} 			% para melhorias de justificação
\usepackage{transparent}
\usepackage{eso-pic}
\usepackage{amsthm,amsfonts}
\usepackage{float}
\usepackage{multirow}
\usepackage[table,xcdraw]{xcolor}
\usepackage{longtable}
\usepackage{lipsum}				% para geração de dummy text
%\usepackage[brazilian,hyperpageref]{backref}	 % Paginas com as citações na bibl
\usepackage[alf]{abntex2cite}	% Citações padrão ABNT
\usepackage{xcolor}
\usepackage{scalefnt}

\usepackage[font=small]{caption}     %% make caption in normal size
\usepackage{etoolbox}
\AtBeginEnvironment{longtabu}{\footnotesize}{}{}   %% change all longtabu content to foot note size

\definecolor{verde}{rgb}{0,0.5,0}
\usepackage{listings}
\lstset{
  language=C++,
  basicstyle=\ttfamily\small,
  keywordstyle=\color{blue},
  stringstyle=\color{verde},
  commentstyle=\color{red},
  extendedchars=true,
  showspaces=false,
  showstringspaces=false,
  numbers=left,
  numberstyle=\tiny,
  breaklines=true,
  backgroundcolor=\color{green!10},
  breakautoindent=true,
  captionpos=b,
  xleftmargin=0pt,
}


\titulo{Processo Agroindustrial de Processamento de Cacau}
\autor{Caio Henrique Silva Souza 99131\\Eduardo Favoretto Vale Bom 108139\\Gabriel Rodrigues Munhoz 106802\\João Vítor Batistão 108074}
\local{Maringá, PR}
\data{01.02.2022}
\orientador{}
\coorientador{}
\instituicao{%
  Universidade Estadual de Maringá - UEM
  \par
  Departamento de Engenharia de Produção - DEP}
\tipotrabalho{Tese (Doutorado)}
\preambulo{}

\definecolor{blue}{RGB}{41,5,195}

\makeatletter
\hypersetup{
     	%pagebackref=true,
	pdftitle={\@title}, 
	pdfauthor={\@author},
    	pdfsubject={\imprimirpreambulo},
	pdfcreator={LaTeX with abnTeX2},
	pdfkeywords={abnt}{latex}{abntex}{abntex2}{trabalho acadêmico}, 
	colorlinks=true,       		% false: boxed links; true: colored links
    	linkcolor=black,          	% color of internal links
    	citecolor=blue,        		% color of links to bibliography
    	filecolor=magenta,      		% color of file links
	urlcolor=blue,
	bookmarksdepth=4
}

\setlength{\parindent}{1.3cm}

\setlength{\parskip}{0.2cm}  % tente também \onelineskip

\makeindex

%\usepackage{fancyhdr}
%\fancyhead{}
%\fancyfoot{}
%\lhead{Processo Agroindustrial de Processamento de Cacau}
%\rhead{Processo Agroindustrial de Processamento de Cacau}

\AddToShipoutPicture{
\put(0,0){
\parbox[b][\paperheight]{\paperwidth}{%
\vfill
\centering
{\transparent{0.1}\includegraphics[scale=1.4]{../../Pictures/logoUEM.jpg}    }%
\vfill}}}



%\graphicspath{{../Pictures}}
\begin{document}

\begin{minipage}[c][0cm][c]{0cm} % a primeira minipágina tem uma altura de 1.5cm e uma largura de 3cm.

\centering

\includegraphics[scale=0.45]{../../Pictures/uem-modelo-04.png}  
\end{minipage}

\selectlanguage{brazil}

\frenchspacing 

% \pretextual

\imprimircapa


% ---
% RESUMOS
% ---

%\setlength{\absparsep}{18pt} % ajusta o espaçamento dos parágrafos do resumo
%\begin{resumo}
 
 
% \textbf{Palavras-chave}: latex. abntex. editoração de texto.
%\end{resumo}


% ---
% inserir o sumario
% ---
\pdfbookmark[0]{\contentsname}{toc}
\tableofcontents*
\cleardoublepage

% ----------------------------------------------------------
% ELEMENTOS TEXTUAIS
% ----------------------------------------------------------
\textual

\chapter{Introdução}
%\pagestyle{fancy}

\section{Contextualização do Tema}

O cacau é um fruto da espécie \textit{Theobroma cacao L.} que possui origem na região amazônica e seu uso, segundo arqueólogos equatorianos e franceses, já era realizado há cerca de 5.500 anos pelos povos amazônicos \cite{2}. No entanto, foi no século XVII que acabou se tornando um produto agrícola e cultivado em diferentes locais da América do Sul e Central devido a disseminação do cultivo pelos espanhóis, e posteriormente se expandindo aos poucos pelo mundo. \cite{1} 

Existem 6 principais produtos a partir do fruto de cacau: mel, polpa, nibs, chocolate, manteiga e cacau em pó, além da própria amêndoa do cacau e casca que também pode ser comercializada de uma forma menos processada. A maior parte desses produtos são voltados para o setor alimentício, no entanto é possível verificar aplicações também no setor cosmético e no setor de geração de energia. \cite{5}

Com mais da metade da produção nacional, 62$\%$, o sul da Bahia é a principal região produtora de cacau, seguida pela região norte do Brasil com 34$\%$ e o restante da produção, 4$\%$, espalhada pelo país \cite{1}. O Brasil é o 7º maior produtor do mundo e segundo o Instituto Brasileiro de Geografia Estatística (IBGE) o Brasil produziu em torno de 310 mil tonaledas. \cite{3} 

\section{Objetivos}

\subsection{Objetivo Geral}

Estudar o processamento de cacau na agroindústria, mais especificamente a produção de chocolate, abrangendo desde a parte normativa do setor, até o entendimento do mercado do produto em questão e as análises técnicas dos processos.

\subsection{Objetivos Específicos}

\begin{itemize}
\item Estudar o mercado, normas regulatórias, oportunidades e desafios do setor.
\item Realizar o mapeamento dos processos de uma produção de chocolate desde o recebimento do fruto até a embalagem, estoque e distribuição do produto final.
\item Calcular os balanços de massa e energia que são inerentes ao processo de produção.
\end{itemize}

\newpage
\chapter{Planejamento Estratégico de Produtos}
%\pagestyle{fancy}

\section{Definir escopo da revisão do PEN}

Nesta atividade do Item “Planejamento estratégico de Produtos”, visto que
não temos uma empresa “real” para o desenvolvimento do produto, portanto,
ela não possui um planejamento estratégico, é preciso definir algumas
questões. Assim, nessa atividade devem ser definidos pelo menos:
Missão
Visão
Valores
Segmento de mercado
Alguns objetivos e metas a serem alcançadas a longo prazo pela empresa
(estratégias).
No mínimo três ideias de produtos para seu portfólio (cada um terá uma
minuta breve, mas somente um será desenvolvido).


\section{Planejar atividades para a revisão do PEN}

Apresentar o planejamento de como se dará as próximas atividades do Item.

\section{Consolidar informações sobre tecnologia e mercado}




\section{Revisar o PEN}


Nesta atividade poderá ser revisto o que foi definido no Item 2.1 pela equipe
com as informações obtidas na atividade 2.3. Apresentando, então, as
possíveis alterações ou a manutenção do que foi previamente definido.

Nesta atividade os produtos definidos na atividade 2.1 podem ser analisados
baseados em técnicas que utilizam critérios relacionados a estratégia da
empresa.

\section{Analisar o portfólio de produtos da empresa}

\section{Propor mudanças no portfólio de produtos}

\section{Verificar a viabilidade do portfólio de produtos}

Aqui fica dispensada a atividade de “Avaliar a viabilidade econômica do
portfólio de projetos”.

\section{Decidir o início do planejamento de um dos produtos do portfólio}

\newpage
\chapter{Planejamento do Projeto}
%\pagestyle{fancy}

\section{Definir interessados do projeto}

\section{Definir escopo do produto}

\section{Definir escopo do projeto}

\section{Detalhar o escopo do projeto}

\section{Adaptar o modelo de referência}

\section{Definir atividades e sequência}

\section{Preparar cronograma}

\section{Avaliar riscos}

\section{Preparar orçamento do projeto}

\section{Analisar a viabilidade econômica do projeto}

\section{Definir indicadores de desempenho}

\section{Definir plano de comunicação}

\section{Planejar e preparar aquisições}

\section{Preparar plano de projeto}


\newpage
\chapter{Projeto Informacional}
%\pagestyle{fancy}

\section{Revisar e atualizar o escopo do produto}

\section{Detalhar ciclo de vida do produto e definir seus clientes}

\section{Identificar os requisitos dos clientes do produto}

\section{Definir os requisitos do produto}

\section{Definir especificações-meta do produto}


\newpage
\chapter{Projeto Conceitual}
%\pagestyle{fancy}

\section{Modelar funcionalmente}

\section{Desenvolver princípios de soluções para as funções}

\section{Desenvolver alternativas de solução}

\section{Definir arquitetura}

\section{Analisar SSCs}

\section{Definir ergonomia e estética}

\section{Definir parcerias de co-desenvolvimento}

\section{Definir plano macro de processo}

\section{Selecionar concepções alternativas}


\newpage
\chapter{Projeto Detalhado}
%\pagestyle{fancy}

\section{Criar e detalhar itens e documentos}

\section{Decidir fazer ou comprar SSCs}

\section{Desenvolver fornecedores}

\section{Planejar processo de fabricação e montagem}

\section{Projetar recursos de fabricação}

\section{Avaliar itens e documentos}

\section{Otimizar produto e processo}

\section{Criar material de suporte do produto}

\section{Projetar embalagem}

\section{Planejar fim de vida de produto}

\section{Testar e homologar produto}


\newpage
\chapter{Prototipagem}
%\pagestyle{fancy}


\newpage
\postextual

\bibliography{referencias}

%\begin{anexosenv}

%\chapter{Código utilizado no software EMSO}

%\begin{lstlisting}
%\end{lstlisting}

%\end{anexosenv}

\end{document}
