%%%%%%%%%%%%%%%%%%%%%%%%%%%%%%%%%%%%%%%%%
% Lachaise Assignment
% LaTeX Template
% Version 1.0 (26/6/2018)
%
% This template originates from:
% http://www.LaTeXTemplates.com
%
% Authors:
% Marion Lachaise & François Févotte
% Vel (vel@LaTeXTemplates.com)
%
% License:
% CC BY-NC-SA 3.0 (http://creativecommons.org/licenses/by-nc-sa/3.0/)
% 
%%%%%%%%%%%%%%%%%%%%%%%%%%%%%%%%%%%%%%%%%

%----------------------------------------------------------------------------------------
%	PACKAGES AND OTHER DOCUMENT CONFIGURATIONS
%----------------------------------------------------------------------------------------

\documentclass{article}

\usepackage{indentfirst}

\input{structure.tex} % Include the file specifying the document structure and custom commands

%----------------------------------------------------------------------------------------
%	ASSIGNMENT INFORMATION
%----------------------------------------------------------------------------------------

\title{Relatório PIBIC/CNPq-FA} % Title of the assignment

\author{Gabriel R. Munhoz\\ Gabriel$\_\_$RM$@$hotmail.com} % Author name and email address

\date{Universidade Estadual de Maringá UEM 04.12.2019} % University, school and/or department name(s) and a date

%----------------------------------------------------------------------------------------


\AddToShipoutPicture{

\put(0,0){

\parbox[b][\paperheight]{\paperwidth}{%

\vfill

\centering

{\transparent{0.1}\includegraphics[scale=1.35]{../../../../../Google Drive/Meu computador/Faculdade/Latex/Trabalhos/TrabBarco/LogoUEM.jpg}  }%

\vfill}}}

\begin{document}

\maketitle % Print the title


%----------------------------------------------------------------------------------------
%	INTRODUCTION
%----------------------------------------------------------------------------------------

\section*{Cronograma} % Unnumbered section

As atividades do plano trabalho desta pesquisa são compostas por um levantamento bibliográfico para desenvolvimento de uma base de artigos científicos, uma correlação de bases de dados disponíveis na área de Segurança e Saúde do Trabalho, assim como a preparação e padronização dos dados selecionados para utilização em softwares, e análise estatística dos mesmos para a elaboração de relatórios e publicações para a apresentação do conhecimento adquirido.


\section{Atividades realizadas} % Numbered section

Durante o período de 1º de agosto à 30 de novembro de 2019 foram realizadas as seguintes atividades: 
\begin{enumerate}[a)]

\item Reuniões em conjunto com o mestrado, 
\item Filtragem de artigos científicos com o auxílio dos mestrandos, 
\item Leitura de diversos artigos científicos,
\item Análise das bases de dados disponíveis,
\item Treinamentos sobre softwares de análise estatística e,
\item Pequenas análises de bancos de dados para treinar e encontrar dados relevantes.

\end{enumerate}

Nessa pesquisa, portanto, existe uma ajuda mútua com o mestrado visulizada nas atividades listadas acima, em que o mestrado consegue passar o conhecimento mais aprofundado de alguns assuntos enquanto a pesquisa de iniciação traz uma ajuda na parte operacional e criativa.

\subsection{Reuniões}

As reuniões ocorreram de forma regular com apresentações tanto de professores quanto de acadêmicos do mestrado. Os temas de cada dia foram sempre vinculados com a área de Segurança e Saúde do Trabalho ou alguma ferramenta de análise estatística, como o software R e o software Weka. 

Tais encontros foram de grande importância para engajamento de toda a equipe de pesquisa e também para agregar novos conhecimentos para todos. 

\subsection{Artigos científicos}

As publicações científicas foram filtradas pelos mestrandos de várias plataformas de literatura revisada e os filtros foram definidos em conjunto com os docentes. 

Após análise minuciosa, ocorreu a leitura dos artigos que ainda perdura para captação de detalhes dos métodos e estratégias definidas por cada autor. 

Tais publicações científicas serão utilizadas como parte da base bibliográfica dos acadêmicos de mestrado e também para montagem da bibliografia das iniciações científicas.

\subsection{Bases de dados de SST}


As principais referências e bases de dados na área de SST que foram definidas durante o período são: 

\begin{enumerate}[a)]

\item Base de dados de acidentes de trabalho notificados (CATWEB), 
\item Base de dados de beneficios previdenciários (SISBEN), 
\item Dados abertos da saúde e segurança do trabalho e outros dados do Ministério da Economia,

\end{enumerate}

Nessas bases de dados é possível encontrar arquivos disponíveis em diversos formatos, no entanto, os dados encontrados normalmente não são padronizados. Assim, é necessário antes de utilizá-los, para análise estatística, fazer a padronização e formatação dos mesmos de acordo com o software que será utilizado.

\subsection{Treinamentos e análises}

Os treinamentos que ocorreram nesse período foram referentes aos softwares de Power BI, Excel, Weka e R que ocorreram em palestras na universidade, apresentações em reuniões e pequenas análises de algumas bases de dados para treinamento. 

Assim, foi possível colocar em prática o conhecimento teórico sobre as ferramentas de análise estatística e já começar a pensar em algo específico para se analisar com tais softwares.

\section{Conclusão}

O cronograma está sendo seguido, porém houveram algumas adaptações por haver uma ajuda dos acadêmicos de mestrado durante o processo que havia sido delimitado. 

É necessário ainda uma leitura profunda de alguns artigos científicos relacionados para analisar qual será o foco da análise estatística e a partir disso selecionar os dados para formatação e padronização. Dessa forma será possível verificar com a ajuda das ferramentas estatísticas alguma tendência ou padrão para ocorrência de acidentes de trabalho.

\newpage
\section*{Novembro}

Em novembro foram realizadas as seguintes atividades:

\begin{enumerate}[1)]

\item Curso de Power BI online - 10h, 
\item Auxílio aos mestrandos - 10h, 
\item Leitura de artigos na área de SST - 4h,
\item Pesquisa em bancos de dados para familiaziração com sites - 10h
\item Pequenas análises de bancos de dados(treinar formatação e visualizar possíveis focos de trabalho) - 4h

\end{enumerate}

%----------------------------------------------------------------------------------------

\end{document}
