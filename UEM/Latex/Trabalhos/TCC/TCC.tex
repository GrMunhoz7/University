\documentclass[
	% -- opções da classe memoir --
	12pt,				% tamanho da fonte
	%openright,			% capítulos começam em pág ímpar (insere página vazia caso preciso)
	oneside,			% para impressão em recto e verso. Oposto a oneside
	a4paper,			% tamanho do papel. 
	% -- opções da classe abntex2 --
	%chapter=TITLE,		% títulos de capítulos convertidos em letras maiúsculas
	%section=TITLE,		% títulos de seções convertidos em letras maiúsculas
	%subsection=TITLE,	% títulos de subseções convertidos em letras maiúsculas
	%subsubsection=TITLE,% títulos de subsubseções convertidos em letras maiúsculas
	% -- opções do pacote babel --
	english,			% idioma adicional para hifenização
	french,				% idioma adicional para hifenização
	spanish,			% idioma adicional para hifenização
	brazil,				% o último idioma é o principal do documento
	]{abntex2}


% ---
% PACOTES
% ---

% ---
% Pacotes fundamentais 
% ---
%\usepackage{fontspec}
\usepackage{pslatex}			% Usa a fonte Latin Modern
\usepackage[T1]{fontenc}		% Selecao de codigos de fonte.
\usepackage[utf8]{inputenc}		% Codificacao do documento (conversão automática dos acentos)
\usepackage{indentfirst}		% Indenta o primeiro parágrafo de cada seção.
\usepackage{color}				% Controle das cores
\usepackage{graphicx}			% Inclusão de gráficos
\usepackage{microtype} 			% para melhorias de justificação
\usepackage{transparent}
\usepackage{eso-pic}
\usepackage{float}
\usepackage{array}
\usepackage{epstopdf}
\usepackage{url}
% ---

% ---
% Pacotes adicionais, usados no anexo do modelo de folha de identificação
% ---
\usepackage{multicol}
\usepackage{multirow}
% ---
	
% ---
% Pacotes adicionais, usados apenas no âmbito do Modelo Canônico do abnteX2
% ---
%\usepackage{lipsum}				% para geração de dummy text
% ---

% ---
% Pacotes de citações
% ---
%\usepackage[brazilian,hyperpageref]{backref}	 % Paginas com as citações na bibl
%\usepackage[alf]{abntex2cite}	% Citações padrão ABNT

% --- 
% CONFIGURAÇÕES DE PACOTES
% --- 

% ---
% Configurações do pacote backref
% Usado sem a opção hyperpageref de backref
%\renewcommand{\backrefpagesname}{Citado na(s) página(s):~}
% Texto padrão antes do número das páginas
%\renewcommand{\backref}{}
% Define os textos da citação
%\renewcommand*{\backrefalt}[4]{
%	\ifcase #1 %
%		Nenhuma citação no texto.%
%	\or
%		Citado na página #2.%
%	\else
%		Citado #1 vezes nas páginas #2.%
%	\fi}%
% ---

% ---
% Informações de dados para CAPA e FOLHA DE ROSTO
% ---
\titulo{Estatística e Probabilidade\\Trabalho sobre Apresentação de Dados\\Parte 2 - Experimental}
\autor{Profª Drª Mara Lúcia Martins Lopes\\Caio da Silva Pereira\\Gabriel Rodrigues Munhoz\\Heitor Martins da Silva\\Leandro Suzuki Barboza dos Santos\\Rafael Gomes de Oliveira\\Thiago Rissetti Roquetto}
\local{Ilha Solteira, São Paulo}
\data{Setembro de 2017}
\instituicao{%
  Universidade Estadual Paulista  - Unesp
  \par
  Faculdade de Engenharia de Ilha Solteira  - FEIS}
\tipotrabalho{Trabalho científico}
% O preambulo deve conter o tipo do trabalho, o objetivo, 
% o nome da instituição e a área de concentração 
\preambulo{"Estatística é método, ciência e arte."}
% ---

% ---
% Configurações de aparência do PDF final

% alterando o aspecto da cor azul
\definecolor{blue}{RGB}{41,5,195}

% informações do PDF
\makeatletter
\hypersetup{
     	%pagebackref=true,
		pdftitle={\@title}, 
		pdfauthor={\@author},
    	pdfsubject={\imprimirpreambulo},
	    pdfcreator={LaTeX with abnTeX2},
		pdfkeywords={abnt}{latex}{abntex}{abntex2}{relatório técnico}, 
		colorlinks=true,       		% false: boxed links; true: colored links
    	linkcolor=blue,          	% color of internal links
    	citecolor=blue,        		% color of links to bibliography
    	filecolor=magenta,      		% color of file links
		urlcolor=blue,
		bookmarksdepth=4
}
\makeatother
% --- 

% --- 
% Espaçamentos entre linhas e parágrafos 
% --- 

% O tamanho do parágrafo é dado por:
\setlength{\parindent}{1.3cm}

% Controle do espaçamento entre um parágrafo e outro:
\setlength{\parskip}{0.2cm}  % tente também \onelineskip

% ---
% compila o indice
% ---
\makeindex
% ---
\usepackage{fancyhdr}
\fancyhead{}
\fancyfoot{}
\lhead{Trabalho sobre Ajustagem Mecânica}
\rhead{\thepage}

%\AddToShipoutPicture{

%\put(0,0){

%\parbox[b][\paperheight]{\paperwidth}{%

%\vfill

%\centering

%{\transparent{0.1}\includegraphics[scale=2]{../../../Imagens/SA03x.jpg}  }%

%\vfill}}}
% ----
% Início do documento
% ----
\begin{document}

%\begin{minipage}[c][1.5cm][c]{3cm} % a primeira minipágina tem uma altura de 1.5cm e uma largura de 3cm.

%\includegraphics[scale=0.6]{../../../Imagens/barraunesp-assvisual.png} 

%\end{minipage}

% Seleciona o idioma do documento (conforme pacotes do babel)
%\selectlanguage{english}
\selectlanguage{brazil}

% Retira espaço extra obsoleto entre as frases.
\frenchspacing 

% ----------------------------------------------------------
% ELEMENTOS PRÉ-TEXTUAIS
% ----------------------------------------------------------
%\pretextual

% ---
% Capa
% ---
\imprimircapa
% ---

% ---
% Folha de rosto
% (o * indica que haverá a ficha bibliográfica)
% ---
\imprimirfolhaderosto*

% ---
% inserir o sumario
% ---
\pdfbookmark[0]{\contentsname}{toc}
\tableofcontents*
\newpage

\chapter*{Agradecimentos:}

Agradeço a Deus, por ter me dado força, sabedoria, paciência e saúde para vencer os desafios diários. Obrigada pelas pessoas que colocou no meu caminho. 

Aos meus pais, que souberam confortar meu cansaço e dar estímulo quando mais precisei em toda a minha vida. Obrigada pelo amor, incentivo e esforço para realizar meus sonhos. Amo vocês. 

À minha irmã Nathalia pela alegria e carinho. Amo você. 
Agradeço a toda minha família, meus avós, tios, padrinhos e primos, pelos momentos de alegria e incentivo. 

Aos queridos e divertidos amigos da residência Lívia, Murillo, Monique, Vanessa, Indianathan, Natália, Talita, Gabriele, Naiara, Bruna e Fernanda pelos momentos de alegrias, angústias, apuros, ajuda e amizade. Foi muito bom conviver com vocês esses anos!

À professora e orientadora Gisleine pelos ensinamentos, dedicação, apoio, competência e amizade durante toda a realização da residência. Seus ensinamentos estarão presentes por toda a minha vida. 

Aos professores Estela, Silvana, Elza, Walderez, Jorge, Terezinha, Nelly e Roselania. Não tenho palavras para descrever a minha gratidão. 

A todos os colaboradores do Hospital Universitário de Maringá, em especial o Serviço de Farmácia Hospitalar, que sempre estavam dispostos a me auxiliar. 

À Maria dos Anjos por me mostrar que aquele que semeia bondade recebe amor. 

Aos preceptores que tive ao longo da residência pelo conhecimento transmitido, paciência e pelas contribuições para a minha vida profissional; 

Aos membros da banca, Dr. Donadio e Me. Simone por gentilmente aceitarem o convite e pelas contribuições. 

A todos que de qualquer forma colaboraram não só para que este trabalho pudesse ser realizado, como também para que eu me transformasse na pessoa que sou.


\section[Objetivo]{Objetivo}
\pagestyle{fancy}



\newpage
\section[Resumo]{Resumo}




\newpage
\section[Procedimento Experimental]{Procedimento Experimental}




% ---
% inserir lista de ilustrações
% ---
%\pdfbookmark[0]{\listfigurename}{lof}
%\listoffigures*
%\cleardoublepage
% ---

% ---
% inserir lista de tabelas
% ---
%\pdfbookmark[0]{\listtablename}{lot}
%\listoftables*
%\cleardoublepage
% ---

% ---
% inserir lista de abreviaturas e siglas
% ---
%\begin{siglas}
% \item[ABNT] Associação Brasileira de Normas Técnicas
%  \item[abnTeX] ABsurdas Normas para TeX
%\end{siglas}
% ---

% ---
% inserir lista de símbolos
% ---
%\begin{simbolos}
  %\item[$ \Gamma $] Letra grega Gama
  %\item[$ \Lambda $] Lambda
  %\item[$ \zeta $] Letra grega minúscula zeta
  %\item[$ \in $] Pertence
%\end{simbolos}
% ---

% ----------------------------------------------------------
% ELEMENTOS TEXTUAIS
% ----------------------------------------------------------
\newpage

\section[Resultados e Discussão]{Resultados e Discussão}
\pagestyle{fancy}



\newpage
\section[Conclusão]{Conclusão}
\pagestyle{fancy}




\end{document}