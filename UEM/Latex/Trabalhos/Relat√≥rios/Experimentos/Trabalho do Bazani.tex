\documentclass[
	% -- opções da classe memoir --
	12pt,				% tamanho da fonte
	%openright,			% capítulos começam em pág ímpar (insere página vazia caso preciso)
	oneside,			% para impressão em recto e verso. Oposto a oneside
	a4paper,			% tamanho do papel. 
	% -- opções da classe abntex2 --
	%chapter=TITLE,		% títulos de capítulos convertidos em letras maiúsculas
	%section=TITLE,		% títulos de seções convertidos em letras maiúsculas
	%subsection=TITLE,	% títulos de subseções convertidos em letras maiúsculas
	%subsubsection=TITLE,% títulos de subsubseções convertidos em letras maiúsculas
	% -- opções do pacote babel --
	english,			% idioma adicional para hifenização
	french,				% idioma adicional para hifenização
	spanish,			% idioma adicional para hifenização
	brazil,				% o último idioma é o principal do documento
	]{abntex2}


% ---
% PACOTES
% ---

% ---
% Pacotes fundamentais 
% ---
\usepackage{pslatex}			% Usa a fonte Latin Modern
\usepackage[T1]{fontenc}		% Selecao de codigos de fonte.
\usepackage[utf8]{inputenc}		% Codificacao do documento (conversão automática dos acentos)
\usepackage{indentfirst}		% Indenta o primeiro parágrafo de cada seção.
\usepackage{color}				% Controle das cores
\usepackage{graphicx}			% Inclusão de gráficos
\usepackage{microtype} 			% para melhorias de justificação
\usepackage{transparent}
\usepackage{eso-pic}
\usepackage{float}
% ---

% ---
% Pacotes adicionais, usados no anexo do modelo de folha de identificação
% ---
\usepackage{multicol}
\usepackage{multirow}
% ---
	
% ---
% Pacotes adicionais, usados apenas no âmbito do Modelo Canônico do abnteX2
% ---
%\usepackage{lipsum}				% para geração de dummy text
% ---

% ---
% Pacotes de citações
% ---
%\usepackage[brazilian,hyperpageref]{backref}	 % Paginas com as citações na bibl
%\usepackage[alf]{abntex2cite}	% Citações padrão ABNT

% --- 
% CONFIGURAÇÕES DE PACOTES
% --- 

% ---
% Configurações do pacote backref
% Usado sem a opção hyperpageref de backref
%\renewcommand{\backrefpagesname}{Citado na(s) página(s):~}
% Texto padrão antes do número das páginas
%\renewcommand{\backref}{}
% Define os textos da citação
%\renewcommand*{\backrefalt}[4]{
%	\ifcase #1 %
%		Nenhuma citação no texto.%
%	\or
%		Citado na página #2.%
%	\else
%		Citado #1 vezes nas páginas #2.%
%	\fi}%
% ---

% ---
% Informações de dados para CAPA e FOLHA DE ROSTO
% ---
\titulo{Estática\\Trabalho sobre Modelo de Vinculação e Reações de Apoio em Suporte}
\autor{Profº Drº Marcio Antonio Bazani\\Bruno Ferro RA: 142054151\\Felipe Perroni Chaim RA: 162054921\\Gabriel Rodrigues Munhoz RA: 162053541\\Jadiel Oliveira Costa RA: 162053681\\Juan Terenciani Batista RA: 162053738\\Matheus de Franca Farias RA: 162054904}
\local{Ilha Solteira, São Paulo}
\data{Setembro de 2017}
\instituicao{%
  Universidade Estadual Paulista  - Unesp
  \par
  Faculdade de Engenharia de Ilha Solteira  - FEIS}
\tipotrabalho{Trabalho científico}
% O preambulo deve conter o tipo do trabalho, o objetivo, 
% o nome da instituição e a área de concentração 
\preambulo{Estudo de um suporte de ar condicionado com seu respectivo modelo de vinculação e reações de apoio existentes.}
% ---

% ---
% Configurações de aparência do PDF final

% alterando o aspecto da cor azul
\definecolor{blue}{RGB}{41,5,195}

% informações do PDF
\makeatletter
\hypersetup{
     	%pagebackref=true,
		pdftitle={\@title}, 
		pdfauthor={\@author},
    	pdfsubject={\imprimirpreambulo},
	    pdfcreator={LaTeX with abnTeX2},
		pdfkeywords={abnt}{latex}{abntex}{abntex2}{relatório técnico}, 
		colorlinks=true,       		% false: boxed links; true: colored links
    	linkcolor=blue,          	% color of internal links
    	citecolor=blue,        		% color of links to bibliography
    	filecolor=magenta,      		% color of file links
		urlcolor=blue,
		bookmarksdepth=4
}
\makeatother
% --- 

% --- 
% Espaçamentos entre linhas e parágrafos 
% --- 

% O tamanho do parágrafo é dado por:
\setlength{\parindent}{1.3cm}

% Controle do espaçamento entre um parágrafo e outro:
\setlength{\parskip}{0.2cm}  % tente também \onelineskip

% ---
% compila o indice
% ---
\makeindex
% ---
\usepackage{fancyhdr}
\fancyhead{}
\fancyfoot{}
\lhead{Trabalho sobre Modelo de Vinculação e Reações de Apoio em Suporte}
\rhead{\thepage}

\AddToShipoutPicture{

\put(0,0){

\parbox[b][\paperheight]{\paperwidth}{%

\vfill

\centering

{\transparent{0.1}\includegraphics[scale=2]{../../Imagens/SA03x.jpg} }%

\vfill}}}
% ----
% Início do documento
% ----
\begin{document}

\begin{minipage}[c][1.5cm][c]{3cm} % a primeira minipágina tem uma altura de 1.5cm e uma largura de 3cm.

\includegraphics[scale=0.6]{../../Imagens/barraunesp-assvisual.png}

\end{minipage}

% Seleciona o idioma do documento (conforme pacotes do babel)
%\selectlanguage{english}
\selectlanguage{brazil}

% Retira espaço extra obsoleto entre as frases.
\frenchspacing 

% ----------------------------------------------------------
% ELEMENTOS PRÉ-TEXTUAIS
% ----------------------------------------------------------
%\pretextual

% ---
% Capa
% ---
\imprimircapa
% ---

% ---
% Folha de rosto
% (o * indica que haverá a ficha bibliográfica)
% ---
%\imprimirfolhaderosto*

% ---
% inserir o sumario
% ---
\pdfbookmark[0]{\contentsname}{toc}
\tableofcontents*
\newpage

\section[Objetivo]{Objetivo}
\pagestyle{fancy}

Analisar as forças de vínculo atuantes em um suporte de filtro de água utilizando diagrama de corpo livre e conceitos de equilíbrio.

\newpage
\section[Desenvolvimento]{Desenvolvimento}

Diversos equipamento precisam ser fixados na parede, como ventiladores, suportes de televisão e ares-condicionados entre outros. Para tal fixação, um aparato simples de metal, muitas vezes, é suficiente. Este trabalho estuda o momento $(torque)$ causado por forças externas em um apoio de filtro de água para bebedouro. \cite{hibbeler2005estatica}

A força externa atuante em nosso sistema é a força Peso do filtro, assumiremos que a massa M de tal filtro é de 200kg, pois há uma capacidade de 180l de água e mais motor e lataria.
 
A representação do suporte do filtro se dá pela \textbf{Figura 1}:

\begin{figure}[H]
\center
\caption{Representação do suporte.}
\includegraphics[scale=0.5]{../../Imagens/Latex/imagem1.jpg} 

\end{figure}

Segue abaixo a representação do filtro, na \textbf{Figura 2}:

\begin{figure}[H]
\center
\caption{Representação do filtro.}
\includegraphics[scale=0.5]{../../Imagens/Latex/imagem2.jpg}  

\end{figure}

\begin{figure}[H]
\center
\caption{Representação do sistema.}
\includegraphics[scale=0.5]{../../Imagens/Latex/imagem3.jpg} 

\end{figure}

Por fim, seguem nas \textbf{Figuras 4, 5 e 6} a cotação das peças do sistema:

\begin{figure}[H]
\center
\caption{Cotação da vista lateral.}
\includegraphics[scale=0.5]{../../Imagens/Latex/imagem4.jpg} 

\end{figure}

\begin{figure}[H]
\center
\caption{Cotação da vista frontal.}
\includegraphics[scale=0.5]{../../Imagens/Latex/imagem5.jpg} 

\end{figure}

\begin{figure}[H]
\center
\caption{Cotação da vista superior.}
\includegraphics[scale=0.35]{../../Imagens/Latex/imagem6.jpg} 

\end{figure}


% ---
% inserir lista de ilustrações
% ---
%\pdfbookmark[0]{\listfigurename}{lof}
%\listoffigures*
%\cleardoublepage
% ---

% ---
% inserir lista de tabelas
% ---
%\pdfbookmark[0]{\listtablename}{lot}
%\listoftables*
%\cleardoublepage
% ---

% ---
% inserir lista de abreviaturas e siglas
% ---
%\begin{siglas}
% \item[ABNT] Associação Brasileira de Normas Técnicas
%  \item[abnTeX] ABsurdas Normas para TeX
%\end{siglas}
% ---

% ---
% inserir lista de símbolos
% ---
%\begin{simbolos}
  %\item[$ \Gamma $] Letra grega Gama
  %\item[$ \Lambda $] Lambda
  %\item[$ \zeta $] Letra grega minúscula zeta
  %\item[$ \in $] Pertence
%\end{simbolos}
% ---

% ----------------------------------------------------------
% ELEMENTOS TEXTUAIS
% ----------------------------------------------------------
\newpage
\section[Resultados]{Resultados}
\pagestyle{fancy}

Atribuiremos a Origem do nosso sistema de coordenadas ao ponto 1, que coincide com o parafuso 1. Desta forma, a posição dos parafusos e do centro de massa do filtro ficam da seguinte forma:

\begin{itemize}

\item Parafuso 1: $(0\widehat{i}+ 0\widehat{j} + 0 \widehat{k})$ m
\item Parafuso 2: $(0,390\widehat{k})$ m
\item Parafuso 3: $(-0,620\widehat{i})$ m
\item Parafuso 4: $(-0,620\widehat{i} + 0,390\widehat{k})$ m
\item Centro de massa do filtro: $(-0,310\widehat{i} + 0,250\widehat{j} + 0,705\widehat{k})$ m

\end{itemize}

Quando a linha de ação da força peso não passa pelo parafuso, este sofre um momento de torção junto com a carga da força. Neste trabalho, analisaremos cada parafuso individualmente, como se toda a força fosse aplicada apenas nele e consideraremos a gravidade como sendo $g=9,8\frac{m}{s^{2}}$. \cite{hibbeler2005estatica}

\subsection{Influência do peso do filtro no parafuso 1}

$\vec{F}$ = $(-1960\widehat{k})$ N

$\vec{r_{1}}$ = $(-0,3325\widehat{i}$ + $0,265\widehat{j}$ + $0,885\widehat{k})$ m

$\vec{M_{1}}$ = $[(-0,3325\widehat{i}$ + $0,265\widehat{j}$ + $0,885\widehat{k})$  x  $(-1960\widehat{k})]$ = $(-522,34\widehat{i}$ – $651,7\widehat{j})$ N.m

M  = 835,196 N.m


\subsection{Influência do peso do filtro no parafuso 2}

$\vec{F}$ = $(-1960\widehat{k})$ N

$\vec{r_{1}}$ = $(-0,3325\widehat{i}$ + $0,265\widehat{j}$ + $0,885\widehat{k})$ m

$\vec{M_{1}}$ = $[(-0,3325\widehat{i}$ + $0,265\widehat{j}$ + $0,885\widehat{k})$  x  $(-1960\widehat{k})]$ = $(-522,34\widehat{i}$ – $651,7\widehat{j})$ N.m

M  = 835,196 N.m

\subsection{Influência do peso do filtro no parafuso 3}

$\vec{F}$ = $(-1960\widehat{k})$ N

$\vec{r_{1}}$ = $(-0,3325\widehat{i}$ + $0,265\widehat{j}$ + $0,885\widehat{k})$ m

$\vec{M_{1}}$ = $[(-0,3325\widehat{i}$ + $0,265\widehat{j}$ + $0,885\widehat{k})$  x  $(-1960\widehat{k})]$ = $(-522,34\widehat{i}$ – $651,7\widehat{j})$ N.m

M  = 835,196 N.m

\subsection{Influência do peso do filtro no parafuso 4}

$\vec{F}$ = $(-1960\widehat{k})$ N

$\vec{r_{1}}$ = $(-0,3325\widehat{i}$ + $0,265\widehat{j}$ + $0,885\widehat{k})$ m

$\vec{M_{1}}$ = $[(-0,3325\widehat{i}$ + $0,265\widehat{j}$ + $0,885\widehat{k})$  x  $(-1960\widehat{k})]$ = $(-522,34\widehat{i}$ – $651,7\widehat{j})$ N.m

M  = 835,196 N.m

Pelo fato de os parafusos estarem a uma mesma distância da linha de ação da força atuante (força peso), pelo princípio de transmissibilidade, o módulo do momento causado em cada parafuso tem o mesmo valor. Isso é percebido ao se ver que o módulo de cada componente do momento em cada parafuso é igual. O princípio da transmissibilidade é descrito da seguinte forma:

\begin{figure}[H]
\center
\caption{Cotação da vista superior.}
\includegraphics[scale=1]{../../Imagens/Latex/imagem7.jpg}  

\end{figure}

\begin{center}
$\vec{F}\bullet\vec{r_{a}} = \vec{F}\bullet\vec{r_{b}}$
\end{center}

O sistema não pode ser mais simplificado, pois há apenas uma força atuante, a qual seria a própria resultante.

\newpage
\section[Conclusão]{Conclusão}
\pagestyle{fancy}

Ao final dos cálculos, concluímos que o princípio de transmissibilidade é válido, que um sistema em que há apenas uma força externa atuando não pode ser ainda mais simplificado e que o momento gerado é suportado em cada um dos parafusos do suporte. O módulo do momento em cada um dos parafusos é de 835,196 N.m. 

\newpage

\bibliographystyle{ieeetr}
\bibliography{REFE}
% ----------------------------------------------------------
% Glossário
% ----------------------------------------------------------
%
% Consulte o manual da classe abntex2 para orientações sobre o glossário.
%
%\glossary

\end{document}
