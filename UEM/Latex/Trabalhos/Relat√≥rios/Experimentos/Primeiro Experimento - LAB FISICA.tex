\documentclass[
	% -- opções da classe memoir --
	12pt,				% tamanho da fonte
	%openright,			% capítulos começam em pág ímpar (insere página vazia caso preciso)
	oneside,			% para impressão em recto e verso. Oposto a oneside
	a4paper,			% tamanho do papel. 
	% -- opções da classe abntex2 --
	%chapter=TITLE,		% títulos de capítulos convertidos em letras maiúsculas
	%section=TITLE,		% títulos de seções convertidos em letras maiúsculas
	%subsection=TITLE,	% títulos de subseções convertidos em letras maiúsculas
	%subsubsection=TITLE,% títulos de subsubseções convertidos em letras maiúsculas
	% -- opções do pacote babel --
	english,			% idioma adicional para hifenização
	french,				% idioma adicional para hifenização
	spanish,			% idioma adicional para hifenização
	brazil,				% o último idioma é o principal do documento
	]{abntex2}


% ---
% PACOTES
% ---

% ---
% Pacotes fundamentais 
% ---
\usepackage{pslatex}			% Usa a fonte Latin Modern
\usepackage[T1]{fontenc}		% Selecao de codigos de fonte.
\usepackage[utf8]{inputenc}		% Codificacao do documento (conversão automática dos acentos)
\usepackage{indentfirst}		% Indenta o primeiro parágrafo de cada seção.
\usepackage{color}				% Controle das cores
\usepackage{graphicx}			% Inclusão de gráficos
\usepackage{microtype} 			% para melhorias de justificação
\usepackage{transparent}
\usepackage{eso-pic}
\usepackage{amsthm,amsfonts}
% ---

% ---
% Pacotes adicionais, usados no anexo do modelo de folha de identificação
% ---
\usepackage{multicol}
\usepackage{multirow}
% ---
	
% ---
% Pacotes adicionais, usados apenas no âmbito do Modelo Canônico do abnteX2
% ---
%\usepackage{lipsum}				% para geração de dummy text
% ---

% ---
% Pacotes de citações
% ---
\usepackage[brazilian,hyperpageref]{backref}	 % Paginas com as citações na bibl
\usepackage[alf]{abntex2cite}	% Citações padrão ABNT

% --- 
% CONFIGURAÇÕES DE PACOTES
% --- 

% ---
% Configurações do pacote backref
% Usado sem a opção hyperpageref de backref
\renewcommand{\backrefpagesname}{Citado na(s) página(s):~}
% Texto padrão antes do número das páginas
\renewcommand{\backref}{}
% Define os textos da citação
\renewcommand*{\backrefalt}[4]{
	\ifcase #1 %
		Nenhuma citação no texto.%
	\or
		Citado na página #2.%
	\else
		Citado #1 vezes nas páginas #2.%
	\fi}%
% ---

% ---
% Informações de dados para CAPA e FOLHA DE ROSTO
% ---
\titulo{Laboratório de Física II\\2º Relatório}
\autor{Gabriel Rodrigues Munhoz\\RA 162053541}
\local{Ilha Solteira, São Paulo}
\data{Abril de 2017}
\instituicao{%
  Universidade Estadual Paulista  - Unesp
  \par
  Faculdade de Engenharia de Ilha Solteira  - FEIS}
\tipotrabalho{Relatório técnico}
% O preambulo deve conter o tipo do trabalho, o objetivo, 
% o nome da instituição e a área de concentração 
\preambulo{Aplicação da Teoria de Erros e Algarismos Significativos em medições feitas com régua e paquímetro, definindo o instrumento mais preciso.}
% ---

% ---
% Configurações de aparência do PDF final

% alterando o aspecto da cor azul
\definecolor{blue}{RGB}{41,5,195}

% informações do PDF
\makeatletter
\hypersetup{
     	%pagebackref=true,
		pdftitle={\@title}, 
		pdfauthor={\@author},
    	pdfsubject={\imprimirpreambulo},
	    pdfcreator={LaTeX with abnTeX2},
		pdfkeywords={abnt}{latex}{abntex}{abntex2}{relatório técnico}, 
		colorlinks=true,       		% false: boxed links; true: colored links
    	linkcolor=blue,          	% color of internal links
    	citecolor=blue,        		% color of links to bibliography
    	filecolor=magenta,      		% color of file links
		urlcolor=blue,
		bookmarksdepth=4
}
\makeatother
% --- 

% --- 
% Espaçamentos entre linhas e parágrafos 
% --- 

% O tamanho do parágrafo é dado por:
\setlength{\parindent}{1.3cm}

% Controle do espaçamento entre um parágrafo e outro:
\setlength{\parskip}{0.2cm}  % tente também \onelineskip

% ---
% compila o indice
% ---
%\makeindex
% ---
\usepackage{fancyhdr}
\fancyhead{}
\fancyfoot{}
\lhead{Relatório sobre 1º Experimento}
\rhead{\thepage}

\AddToShipoutPicture{

\put(0,0){

\parbox[b][\paperheight]{\paperwidth}{%

\vfill

\centering

{\transparent{0.1}\includegraphics[scale=2]{../../../../Pictures/SA03x.jpg} }%

\vfill}}}
% ----
% Início do documento
% ----
\begin{document}

\begin{minipage}[c][1.5cm][c]{3cm} % a primeira minipágina tem uma altura de 1.5cm e uma largura de 3cm.

\includegraphics[scale=0.6]{../../../../Pictures/barraunesp-assvisual.png}

\end{minipage}

% Seleciona o idioma do documento (conforme pacotes do babel)
%\selectlanguage{english}
\selectlanguage{brazil}

% Retira espaço extra obsoleto entre as frases.
\frenchspacing 

% ----------------------------------------------------------
% ELEMENTOS PRÉ-TEXTUAIS
% ----------------------------------------------------------
% \pretextual

% ---
% Capa
% ---
\imprimircapa
% ---

% ---
% Folha de rosto
% (o * indica que haverá a ficha bibliográfica)
% ---
%\imprimirfolhaderosto*

% ---
% inserir o sumario
% ---
\pdfbookmark[0]{\contentsname}{toc}
\tableofcontents*
\newpage

\section[Objetivo]{Objetivo}
\pagestyle{fancy}
Determinar as equações do deslocamento, velocidade e aceleração angular de uma esfera rolante num plano inclinado 
\newpage
\section[Resumo]{Resumo}
\pagestyle{fancy}
O experimento realizado consistiu em determinar equações do deslocamento, velocidade e aceleração para um corpo que se deslocava em um plano inclinado com atrito. Após realizado todos os cálculos verificamos uma diferença de momento de inércia de 272$\%$ e uma diferença de energia mecânica final e inicial de 45$\%$. Os erros foram colocados como sendo causados principalmente pelo atrito e pela velocidade de reação humana na demarcação do tempo, pois foi se utilizado cronômetro manual.
% ---
% inserir lista de ilustrações
% ---
%\pdfbookmark[0]{\listfigurename}{lof}
%\listoffigures*
%\cleardoublepage
% ---
% ---
% inserir lista de tabelas
% ---
%\pdfbookmark[0]{\listtablename}{lot}
%\listoftables*
%\cleardoublepage
% ---

% ---
% inserir lista de abreviaturas e siglas
% ---
%\begin{siglas}
% \item[ABNT] Associação Brasileira de Normas Técnicas
%  \item[abnTeX] ABsurdas Normas para TeX
%\end{siglas}
% ---

% ---
% inserir lista de símbolos
% ---
%\begin{simbolos}
  %\item[$ \Gamma $] Letra grega Gama
  %\item[$ \Lambda $] Lambda
  %\item[$ \zeta $] Letra grega minúscula zeta
  %\item[$ \in $] Pertence
%\end{simbolos}
% ---

% ----------------------------------------------------------
% ELEMENTOS TEXTUAIS
% ----------------------------------------------------------
\newpage
\section[Introdução Teórica]{Introdução Teórica}
\pagestyle{fancy}
\subsection[Rolamento]{Rolamento} 
Para uma melhor analise do movimento de rotação , tomaremos como exemplo  uma roda de bicicleta.
Quando uma roda move-se sobre uma pista plana, seu centro de massa desloca-se para frente em um movimente de translação. 
Considere um roda de raio R, rolando sem deslizar em uma superfície plana. Quando a roda gira um ângulo $\theta$, o ponto de contato do aro com a superfície  horizontal se desloca uma distancia s, tal que:

S= R*$\theta$ (1)

O centro de massa da roda deslocou-se a mesma distância, assim a velocidade do centro de massa (Vcm) pode ser escrita da seguinte forma:

Vcm=ds/dt  (2)

A velocidade angular $\omega$ em torno do centro pode ser representada da seguinte forma:

$\omega$= d$\theta$/dt  (3) 

Assim derivando a equação (1) em função do tempo, e adotando que R é uma constante , temos:

Vcm= $\omega$*R  (4)

\subsection[Energia cinética de rolamento]{Energia cinética de rolamento} 
Para calcular a energia cinética de roda em movimento , considere um  ponto P na roda. A energia cinética é dada por:

K= 1/2 * Ip $\omega^{2}$  (5)

Onde $\omega$ é o modulo da velocidade angular e Ip é o momento de inercia em relação ao ponto P. Pelo teorema de eixos paralelos temos:

Ip= Icm + M * R$^{2}$  (6)

Na qual M e a massa da roda e Icm o momento de inercia  do centro de massa.
Para uma esfera, o Icm e representado da seguinte forma:

Icm= (2 * M * R$^{2}$)/5  (7)

Assim a energia cinética  pode ser representada da seguinte forma :

K= (M * Vcm + Icm$\omega^{2}$)/2 (8)

% ----------------------------------------------------------
% PARTE - preparação da pesquisa
% ----------------------------------------------------------
\newpage
\section[Procedimento Experimental]{Procedimento Experimental}
\pagestyle{fancy}
\subsection[Materiais]{Materiais}
Nesse experimento foi usado os seguinte materiais:

-esfera de aço;

-trilho com marcações;

-trena e régua;

-cronômetro;

-paquímetro;

-balança semi-analitica;

\subsection[Método]{Método} 
Primeiramente o trilho foi dividido em oito partes iguais, cada uma com 10 cm de comprimento. O trilho foi colocado em uma posição inclinada, apoiada em um suporte de madeira e medimos a altura da barra e a distância do suporte de madeira até o fim da barra, e obtivemos o valores de 10 cm e 89,2cm ,respecrivamente. 

Com isso, soubemos que o ângulo de inclinação formado era de $\theta$= 6,4 graus. Em seguida medimos o diâmetro da esfera (d=0,031m) e também a pesamos (M= 0,111kg).

A seguir a esfera foi solta do início  do trilho, e foram feitas, com o auxílio do celular, as medições de tempo da esfera ao se deslocar por cada marcação. Após realizado isso foram feitas as marcações do tempo nos gráficos e calculadas as equações de deslocamento, velocidade e aceleração

% ----------------------------------------------------------
% Capitulo com exemplos de comandos inseridos de arquivo externo 
% ----------------------------------------------------------

% ----------------------------------------------------------
% Parte de revisãod e literatura
% ----------------------------------------------------------
\newpage
\section[Resultados e Discussão]{Resultados e Discussão}
\pagestyle{fancy}
A partir do gráfico espaço em função do tempo no papel dilog, obtivemos a coeficiente linear onde o gráfico cortava o eixo das coordenas: 

a=0,38
 
E por meio de cálculos conseguimos o coeficiente angular:

n= (logyf  - logyi) / (logxf  - logxi)

n= (log0,45 - log0,10) / (log1,1 - log0,45)

n= 1,68

Portanto a equação horaria (S x t): 

S= 0,38t$^{1,68}$

Por meio da derivada dessa função obtemos a equação da velocidade:

V=0,64t$^{0,68}$

E a partir da derivada segunda da função horaria obtivemos a equação da aceleração:

a=0,43t$^{-0,32}$

Utilizando a equação (4) adquirimos a equação da velocidade angular:

0,64t$^{0,68}$=$\omega$ * 0,0155

$\omega$= 41,3t$^{0,68}$

E derivando essa equação para encontrarmos a aceleração angular temos:

$\alpha$= 28,1t$^{-0,32}$

Substituindo o valor do tempo no instante B na equação da velocidade angular, temos que:

$\omega$= 41,3 * 1,6$^{0,68}$

$\omega$= 56.8rad/s

Pela formula (7) calculamos o momento de inercia que a esfera deveria apresentar teoricamente:

Icm= (2 * 0,111 * 0,0155$^{2}$)/5

Icm= 1,1 * 10$^{-5}$

Porém, experimentalmente calculamos o momento de inercia através da equação da energia mecânica. Sabendo que a energia mecânica inicial, segundo a teoria, deve ser igual a energia mecânica final, tiramos que:

Em1= Em2

Ep1 + Ec1 = Ep2 + Ec2

M*g*h = K

A partir da formula (8):

K= $\omega^{2}$ (Icm + m*R$^{2}$)/2

M*g*h = $\omega^{2}$ (Icm + mR$^{2}$)/2

0,111 * 9,8 * 0,1 = 56,8$^{2}$ (Icm + 0,111 * 0,0155$^{2}$)/2

Icm = 4,1*10$^{-5}$

Para verificarmos se a energia se conserva calculamos a energia mecânica inicial e a final:

Em1 = Ep1 + Ec1

Em1 = m * g * h

Em1 = 0,111 * 9,8 * 0,1

Em1 = 0,11 J

Em2 = Ep2 + Ec2

Em2 = $\omega^{2}$ (M * R$^{2}$ + Icm )/2

Em2 = 56,8$^{2}$ (0,111 * 0,0155$^{2}$ + 1,1 * 10$^{-5}$) / 2

Em2 = 0,06J

Tabela com Resultados!

Colocar as equações encontradas do deslocamento, velocidade, aceleração linear, velocidade e aceleração angular e comparar valores de momento de inercia e energia mecanica

A partir dos dados obtidos concluímos que houve uma perda de energia mecânica no sistema e com isso o momento de inercia aumentou. As causas para esse acontecimento foram principalmente o atrito e tempo de reação humana por termos utilizado cronometro manual. E as medidas também podem ser considerados como não exatas.
% ---
% ---
\newpage
% ---
\section[Conclusão]{Conclusão}
\pagestyle{fancy}% ---
O experimento realizado demonstrou um movimento variável de acordo com as equações:

S= 0,38t$^{1,68}$ 

V= 0,64t$^{0,68}$

a= 0,43t$^{-0,32}$

$\omega$= 41,3t$^{0,68}$

$\alpha$ = 28,1t$^{-0,32}$

E ao compararmos o momento de inercia teórico com o experimental notamos um erro percentual de: 272$\%$

E ao observarmos a conservação na energia vimos uma perde de 0,05J gerando um erro percentual em comparação com a teoria 
de: 45$\%$

\postextual

% ----------------------------------------------------------
% Referências bibliográficas
% ----------------------------------------------------------
\newpage
\section[Referências Bibliográficas]{Referências Bibliográficas}
\pagestyle{fancy}

1) Apostila de Laboratório de Física, 2008

2) Halliday $\&$ Resnick, Fundamentos de Física, vol. 1
% ----------------------------------------------------------
% Glossário
% ----------------------------------------------------------
%
% Consulte o manual da classe abntex2 para orientações sobre o glossário.
%
%\glossary

\end{document}
